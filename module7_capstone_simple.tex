\documentclass[12pt]{article}
\usepackage[utf8]{inputenc}
\usepackage[margin=1in]{geometry}
\usepackage{color}
\usepackage{titlesec}
\usepackage{times}
\usepackage{graphicx}

% Define section and subsection styles to match the image
\titleformat{\section}{\color{blue}\Large\bfseries}{\thesection}{1em}{}
\titleformat{\subsection}{\large\bfseries}{\thesubsection}{1em}{}
\titleformat{\subsubsection}{\normalsize\bfseries}{\thesubsubsection}{1em}{}

\begin{document}

% Cover Page
\begin{titlepage}
\centering

% MSOE Logo
\includegraphics[width=3cm]{LOGO.png}\\[1cm]

% Main Title
{\Huge\bfseries\color{blue} GAMBIAN RIGHTS AI ASSISTANT}\\[0.5cm]

% Subtitle
{\Large Module 7 Capstone Assignment}\\[2cm]

% Project Details
\textbf{Project:} AI-Powered Legal Rights Chatbot for The Gambia\\[0.5cm]
\textbf{Team Member:} Alhagie Boye\\[0.5cm]
\textbf{Course:} AI for Business Capstone Project\\[0.5cm]

% Report Date
\textbf{Report Date:} \today

\end{titlepage}

% Start of actual content
\newpage

\section{Part 1 – Problem Identification \& Goal Framing (25 Points)}

\subsection{Problem Statement}

\subsubsection{The Issue}
In The Gambia, citizens face a critical knowledge gap regarding their fundamental rights and legal protections. There is no centralized, easily accessible source for citizens to understand their constitutional rights, legal obligations, and available remedies when their rights are violated. This lack of awareness leaves Gambians vulnerable to exploitation, unable to advocate for themselves, and disconnected from the legal protections that should safeguard their daily lives.

\subsubsection{Why It Matters}
The absence of accessible legal information in The Gambia creates a significant barrier to justice and civic engagement. Many citizens are unaware of their basic rights regarding employment, property, family law, and civil liberties. This knowledge gap perpetuates inequality, as those without legal knowledge cannot effectively navigate disputes, access government services, or hold institutions accountable. The current system relies heavily on expensive legal consultations that are unaffordable for most citizens, creating a justice system that serves only those with financial means.

\subsubsection{How AI Could Enhance or Automate It}
AI technology presents a transformative opportunity to democratize legal knowledge in The Gambia through an intelligent chatbot system. By integrating OpenAI's advanced language processing capabilities with Gambian legal texts, constitutional documents, and rights-based resources, we can create an accessible legal assistant that speaks in simple, understandable language. The chatbot would be trained on Gambian laws, constitutional rights, and legal procedures, enabling citizens to ask questions in their own words and receive clear, actionable information about their rights and legal options. This system would be accessible via mobile phones and basic internet connections, breaking down traditional barriers to legal information.

\subsubsection{Desired Outcome or Success Measure}
The desired outcome is a functional prototype AI-powered legal chatbot that demonstrates the feasibility of providing accessible legal information to Gambian citizens. As a proof-of-concept, success will be measured by the chatbot's ability to accurately process and respond to basic legal queries about rights and laws (target: 80\% accuracy), user comprehension of responses (target: 3.5/5.0 for clarity), and successful demonstration of the core functionality with a small test group (target: 20-50 test users). This prototype will prove the concept works and can be scaled up to serve the broader Gambian population, ultimately empowering citizens with knowledge of their rights.

\end{document}
